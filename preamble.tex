%% Preambulo de la tesis
\usepackage[utf8]{inputenc}
\usepackage[spanish]{babel}
\usepackage{graphicx}
\usepackage{epsfig}
\usepackage{subfigure}
\usepackage{wrapfig}
\usepackage{lscape}
\usepackage{rotating}
\usepackage{mathrsfs,amsmath}   % need for subequations
\usepackage{mathtools}
\usepackage{amssymb}
\usepackage{color}	
\usepackage{emptypage}
\usepackage{multirow}
\usepackage{dirtytalk}
\usepackage{soul}
%\usepackage{ulem}
\decimalpoint
%\usepackage[latin1]{inputenc}%para las tildes
%\usepackage{charter}%font
\usepackage{scalefnt}		%agranda la letra del texto no los ttulos va con el comando \salefont{0.5} ms abajo
\usepackage{epstopdf}

\usepackage{extsizes} 		%permite tamaos 9 8 14 17 20
\usepackage{fancyhdr} 		%fancy header abajo est la configuracin.
\usepackage{comment}
\usepackage{appendix}
\usepackage{float}

\usepackage{acronym}

\usepackage{titlesec}
\usepackage{epigraph}
\setlength\epigraphwidth{.6\textwidth}
\setlength\epigraphrule{0.25pt}
%\usepackage[Sonny]{fncychap}

\usepackage{caption}
\captionsetup{margin=1cm}
%\captionsetup{width=.95\textwidth}

\usepackage{hyperref}
\hypersetup{
	colorlinks,%
	citecolor=black,%
	filecolor=black,%
	linkcolor=black,%
	urlcolor=black
}

%%%-------------------------------------------
% Paquete para blintext
\usepackage{lipsum}
\usepackage[pangram]{blindtext}
\usepackage{algorithm}

%%-----------Bibliography

%\usepackage[sorting=none,backend=bibtex]{biblatex} %Imports biblatex package
%\addbibresource{sample.bib} %Import the bibliography file
\usepackage[style=alphabetic,backend=bibtex]{biblatex}
%\usepackage[style=alphabetic]{biblatex}
\addbibresource{sample.bib} %Imports bibliography file
%%-----------Utilizar colores para texto

%%----------- Más herramientas de matemática
\usepackage{chemformula}
%%----------- Cosas de química
\usepackage{chemist}


% \titleformat{\chapter}[display]
%   {\bfseries\Large}
%   {\filright{\chaptertitlename} \Huge\thechapter}
%   {1ex}
%   {\titlerule\vspace{1ex}\filright}
%   [\vspace{1ex}\titlerule]


%\input{Qcircuit.tex}
%%%%%%%%%%%%%%%%%%%%%%%%%%%%%%%%%%
%Pgina y Mrgenes
%%%%%%%%%%%%%%%%%%%%%%%%%%%%%%%%%%

% a4 paper tiene 210mm*297mm
% a5 paper tiene 148.5mm*210mm
%\voffset=-2cm \hoffset=-2 cm
%\linespread{1.5}
%\hyphenation{}
%\topskip=0cm
%\headsep=0cm

\usepackage[a4paper,width=160mm,top=25mm,bottom=25mm,bindingoffset=15mm]{geometry}

\begin{comment}
\textwidth      = 160 mm
\textheight     = 230 mm
\oddsidemargin  = 0.0 cm 	%0 equivale a est a 1in o 2.54cm del borde
\evensidemargin = 0.0 cm
\topmargin      = 0   cm	%
\headheight     = 13pt
\end{comment}


%% Fancy Header %%%%%%%%%%%%%%%%%%%%%%%%%%%%%%%%%%%%%%%%%%%%%%%%%%%%%%%%%%%%%%%%%%
% Fancy Header Style Options

\pagestyle{fancy}                       		% Sets fancy header and footer
\fancyfoot{}                            		% Delete current footer settings
\renewcommand{\chaptermark}[1]{         		% Lower Case Chapter marker style
	\markboth{\chaptername\ \thechapter.\ #1}{}}	%
\renewcommand{\sectionmark}[1]{         		% Lower case Section marker style
	\markright{\thesection.\ #1}}         			%
\fancyhead[LE,RO]{\small\thepage}    			% Page number (boldface) in left on even  pages and right on odd pages
\fancyhead[RE]{\small\leftmark}      			% Chapter in the right oneven pages
\fancyhead[LO]{\small\rightmark}     			% Section in the left on odd pages

\renewcommand{\headrulewidth}{1pt}    			% Width of head rule

%%% Clear Header %%%%%%%%%%%%%%%%%%%%%%%%%%%%%%%%%%%%%%%%%%%%%%%%%%%%%%%%%%%%%%%%%%






%%%%%%%%%%%%%%%%%%%%%%%%%%%%%%%%%%%%%%%%%%%%%%%5
%Nuevos Comandos
%%%%%%%%%%%%%%%%%%%%%%%%%%%%%%%%%%%%%%%%%%%%%%55
%hace ecuaciones comentadas como en el libro de dirac.
\newcommand{\eccom}[2]{
	\begin{minipage}{4cm}
		#1
	\end{minipage}
	\begin{minipage}[b]{10cm}
		\begin{eqnarray}
		#2
		\end{eqnarray}
	\end{minipage}
	\newline}


%hace daggers
\newcommand{\da}{^\dagger}
%hace brakets
\newcommand{\ket}[1]{| #1\rangle}
\newcommand{\bra}[1]{\langle #1|}
\newcommand{\braket}[2]{\langle#1\vert#2\rangle}
\newcommand{\ketbra}[2]{\vert#1\rangle \langle#2\vert}
\newcommand{\mean}[1]{\langle #1\rangle}
\newcommand{\abs}[1]{\bigl| #1 \bigr|}

\newcommand{\su}{\uparrow}
\newcommand{\sd}{\downarrow}

\newcommand{\E}{\mathcal{E}}
%\newcommand{\trace}[1]{\textsf{Tr}(#1)}
\newcommand{\trace}[1]{\mbox{Tr}\left( #1 \right)}
%\newcommand{\trace}[0]{\operatorname{tr}}

\newcommand{\beqa}{\begin{eqnarray}}
\newcommand{\eeqa}{\end{eqnarray}}
\newcommand{\beq}{\begin{equation}}
\newcommand{\eeq}{\end{equation}}

\newcommand{\lau}[1]{{\color{red} #1}}

%\newcommand{\comentario}[1]{{\color{red} #1}}
\newcommand{\comentario}[1]{}
\renewcommand{\spanishtablename}{\footnotesize{Tabla}}
\renewcommand{\baselinestretch}{1.5}

%para que ponga ms o menos info en la tabla de contenidos
\setcounter{tocdepth}{2}
\setcounter{secnumdepth}{2}
\setcounter{chapter}{0}
\setlength{\parskip}{1em}

\graphicspath{{/home/daniu/Documentos/tesiiii_daniu/figuras/}}
\graphicspath{{/home/daniu/Documents/figuras_tesis/}}

%%%%%%%%%%%%%%%%%%%%%%%%%%%%%%%%%%%%%%%%%%%%55
%% agregado por daniu

\titleformat{\chapter}[display]
{\normalfont\large\bfseries}{\chaptertitlename\ \thechapter}{0pt}{\Large}

% this alters "before" spacing (the second length argument) to 0
\titlespacing*{\chapter}{0pt}{0pt}{40pt}

\titleformat{\section}
{\normalfont\large\bfseries}{\thesection}{1em}{}
\titlespacing*{\section}{0pt}{20pt}{0pt}

\titleformat{\subsection}
{\normalfont\large\bfseries}{\thesubsection}{1em}{}

\titlespacing*{\subsection}{0pt}{20pt}{0pt}

\renewcommand{\chaptermark}[1]{         		% Lower Case Chapter marker style
	\markboth{\chaptername\ \thechapter.}{}}	

\renewcommand{\sectionmark}[1]{         		% Lower case Section marker style
	\markright{\thesection.}}         			%
\fancyhead[LO]{\small\leftmark}     			% Section in the left on odd pages

\newcommand{\repeatcaption}[2]{%
	\renewcommand{\thefigure}{\ref{#1}}%
	\captionsetup{list=no}%
	\caption{#2}%
}