\documentclass[a4paper,12pt,twoside]{book}
    
%%%%%%%%%%%%%%%%%%%%%%%%%%%%%%%%%%%%%%%%%%%
%% Preámbulo del archivo    
%%%%%%%%%%%%%%%%%%%%%%%%%%%%%%%%%%%%%%%%%%%
%% Preambulo de la tesis
\usepackage[utf8]{inputenc}
\usepackage[spanish]{babel}
\usepackage{graphicx}
\usepackage{epsfig}
\usepackage{subfigure}
\usepackage{wrapfig}
\usepackage{lscape}
\usepackage{rotating}
\usepackage{mathrsfs,amsmath}   % need for subequations
\usepackage{mathtools}
\usepackage{amssymb}
\usepackage{color}	
\usepackage{emptypage}
\usepackage{multirow}
\usepackage{dirtytalk}
\usepackage{soul}
%\usepackage{ulem}
\decimalpoint
%\usepackage[latin1]{inputenc}%para las tildes
%\usepackage{charter}%font
\usepackage{scalefnt}		%agranda la letra del texto no los ttulos va con el comando \salefont{0.5} ms abajo
\usepackage{epstopdf}

\usepackage{extsizes} 		%permite tamaos 9 8 14 17 20
\usepackage{fancyhdr} 		%fancy header abajo est la configuracin.
\usepackage{comment}
\usepackage{appendix}
\usepackage{float}

\usepackage{acronym}

\usepackage{titlesec}
\usepackage{epigraph}
\setlength\epigraphwidth{.6\textwidth}
\setlength\epigraphrule{0.25pt}
%\usepackage[Sonny]{fncychap}

\usepackage{caption}
\captionsetup{margin=1cm}
%\captionsetup{width=.95\textwidth}

\usepackage{hyperref}
\hypersetup{
	colorlinks,%
	citecolor=black,%
	filecolor=black,%
	linkcolor=black,%
	urlcolor=black
}

%%%-------------------------------------------
% Paquete para blintext
\usepackage{lipsum}
\usepackage[pangram]{blindtext}
\usepackage{algorithm}

%%-----------Bibliography

%\usepackage[sorting=none,backend=bibtex]{biblatex} %Imports biblatex package
%\addbibresource{sample.bib} %Import the bibliography file
\usepackage[style=alphabetic,backend=bibtex]{biblatex}
%\usepackage[style=alphabetic]{biblatex}
\addbibresource{sample.bib} %Imports bibliography file
%%-----------Utilizar colores para texto

%%----------- Más herramientas de matemática
\usepackage{chemformula}
%%----------- Cosas de química
\usepackage{chemist}


% \titleformat{\chapter}[display]
%   {\bfseries\Large}
%   {\filright{\chaptertitlename} \Huge\thechapter}
%   {1ex}
%   {\titlerule\vspace{1ex}\filright}
%   [\vspace{1ex}\titlerule]


%\input{Qcircuit.tex}
%%%%%%%%%%%%%%%%%%%%%%%%%%%%%%%%%%
%Pgina y Mrgenes
%%%%%%%%%%%%%%%%%%%%%%%%%%%%%%%%%%

% a4 paper tiene 210mm*297mm
% a5 paper tiene 148.5mm*210mm
%\voffset=-2cm \hoffset=-2 cm
%\linespread{1.5}
%\hyphenation{}
%\topskip=0cm
%\headsep=0cm

\usepackage[a4paper,width=160mm,top=25mm,bottom=25mm,bindingoffset=15mm]{geometry}

\begin{comment}
\textwidth      = 160 mm
\textheight     = 230 mm
\oddsidemargin  = 0.0 cm 	%0 equivale a est a 1in o 2.54cm del borde
\evensidemargin = 0.0 cm
\topmargin      = 0   cm	%
\headheight     = 13pt
\end{comment}


%% Fancy Header %%%%%%%%%%%%%%%%%%%%%%%%%%%%%%%%%%%%%%%%%%%%%%%%%%%%%%%%%%%%%%%%%%
% Fancy Header Style Options

\pagestyle{fancy}                       		% Sets fancy header and footer
\fancyfoot{}                            		% Delete current footer settings
\renewcommand{\chaptermark}[1]{         		% Lower Case Chapter marker style
	\markboth{\chaptername\ \thechapter.\ #1}{}}	%
\renewcommand{\sectionmark}[1]{         		% Lower case Section marker style
	\markright{\thesection.\ #1}}         			%
\fancyhead[LE,RO]{\small\thepage}    			% Page number (boldface) in left on even  pages and right on odd pages
\fancyhead[RE]{\small\leftmark}      			% Chapter in the right oneven pages
\fancyhead[LO]{\small\rightmark}     			% Section in the left on odd pages

\renewcommand{\headrulewidth}{1pt}    			% Width of head rule

%%% Clear Header %%%%%%%%%%%%%%%%%%%%%%%%%%%%%%%%%%%%%%%%%%%%%%%%%%%%%%%%%%%%%%%%%%






%%%%%%%%%%%%%%%%%%%%%%%%%%%%%%%%%%%%%%%%%%%%%%%5
%Nuevos Comandos
%%%%%%%%%%%%%%%%%%%%%%%%%%%%%%%%%%%%%%%%%%%%%%55
%hace ecuaciones comentadas como en el libro de dirac.
\newcommand{\eccom}[2]{
	\begin{minipage}{4cm}
		#1
	\end{minipage}
	\begin{minipage}[b]{10cm}
		\begin{eqnarray}
		#2
		\end{eqnarray}
	\end{minipage}
	\newline}


%hace daggers
\newcommand{\da}{^\dagger}
%hace brakets
\newcommand{\ket}[1]{| #1\rangle}
\newcommand{\bra}[1]{\langle #1|}
\newcommand{\braket}[2]{\langle#1\vert#2\rangle}
\newcommand{\ketbra}[2]{\vert#1\rangle \langle#2\vert}
\newcommand{\mean}[1]{\langle #1\rangle}
\newcommand{\abs}[1]{\bigl| #1 \bigr|}

\newcommand{\su}{\uparrow}
\newcommand{\sd}{\downarrow}

\newcommand{\E}{\mathcal{E}}
%\newcommand{\trace}[1]{\textsf{Tr}(#1)}
\newcommand{\trace}[1]{\mbox{Tr}\left( #1 \right)}
%\newcommand{\trace}[0]{\operatorname{tr}}

\newcommand{\beqa}{\begin{eqnarray}}
\newcommand{\eeqa}{\end{eqnarray}}
\newcommand{\beq}{\begin{equation}}
\newcommand{\eeq}{\end{equation}}

\newcommand{\lau}[1]{{\color{red} #1}}

%\newcommand{\comentario}[1]{{\color{red} #1}}
\newcommand{\comentario}[1]{}
\renewcommand{\spanishtablename}{\footnotesize{Tabla}}
\renewcommand{\baselinestretch}{1.5}

%para que ponga ms o menos info en la tabla de contenidos
\setcounter{tocdepth}{2}
\setcounter{secnumdepth}{2}
\setcounter{chapter}{0}
\setlength{\parskip}{1em}

\graphicspath{{/home/daniu/Documentos/tesiiii_daniu/figuras/}}
\graphicspath{{/home/daniu/Documents/figuras_tesis/}}

%%%%%%%%%%%%%%%%%%%%%%%%%%%%%%%%%%%%%%%%%%%%55
%% agregado por daniu

\titleformat{\chapter}[display]
{\normalfont\large\bfseries}{\chaptertitlename\ \thechapter}{0pt}{\Large}

% this alters "before" spacing (the second length argument) to 0
\titlespacing*{\chapter}{0pt}{0pt}{40pt}

\titleformat{\section}
{\normalfont\large\bfseries}{\thesection}{1em}{}
\titlespacing*{\section}{0pt}{20pt}{0pt}

\titleformat{\subsection}
{\normalfont\large\bfseries}{\thesubsection}{1em}{}

\titlespacing*{\subsection}{0pt}{20pt}{0pt}

\renewcommand{\chaptermark}[1]{         		% Lower Case Chapter marker style
	\markboth{\chaptername\ \thechapter.}{}}	

\renewcommand{\sectionmark}[1]{         		% Lower case Section marker style
	\markright{\thesection.}}         			%
\fancyhead[LO]{\small\leftmark}     			% Section in the left on odd pages

\newcommand{\repeatcaption}[2]{%
	\renewcommand{\thefigure}{\ref{#1}}%
	\captionsetup{list=no}%
	\caption{#2}%
}

%%%%%%%%%%%%%%%%%%%%%%%%%%%%%%%%%%%%%%%%%%%
%% Arranca el documento
%%%%%%%%%%%%%%%%%%%%%%%%%%%%%%%%%%%%%%%%%%%
\begin{document}

\frontmatter

%%%%%%%%%%%%%%%%%%%%%%%%%%%%%%%%%%%%%%%%%%%%%%%%%%%%
%% Caratula
%%%%%%%%%%%%%%%%%%%%%%%%%%%%%%%%%%%%%%%%%%%%%%%%%%%%

\thispagestyle{empty}

\begin{center} 
	\large{
		\epsfig{file=logo_fcen_uba.png, angle=0, width=0.25\textwidth}\vspace{1cm}\\
		\textbf{UNIVERSIDAD DE BUENOS AIRES}\\
		Facultad de Ciencias Exactas y Naturales\\
		Departamento de Ciencias de la Atmósfera y los Océanos\vspace{1.25cm}\\
		\textbf{\LARGE Las tendencias de largo plazo de la temperatura superficial del mar alrededor de	Sudamérica y su posible impacto ecológico} 
		\vspace{0.5cm}\\ 
		Tesis presentada para optar al título de \\
		Doctor de la Universidad de Buenos Aires en el área Ciencias de la Atmósfera y los Océanos\\
		por \textbf{Lic. Daniela B. Risaro} \vspace{1.25cm}\\ 
		Director de Tesis: Alberto R. Piola\\
		Co-Directora de Tesis: Dra. María Paz Chidichimo\\
		Consejero de estudios: Dr. Martín Saraceno  \\
		Lugar de trabajo: Departamento de Oceanografía, Servicio de Hidrografía Naval (SHN)
		\vspace{1.25cm}\\ 
		Buenos Aires, 2020
	}
\end{center}


\chapter*{Resumen}

\addcontentsline{toc}{chapter}{Resumen}
\vspace{-1cm}
\begin{center}
	\textbf{\large{Las pelotudez mas grande del mundo}}
\end{center}
\input{resumen_esp.tex}

\chapter*{Abstract}
\vspace{-1cm}
\begin{center}
\textbf{\large{The biggest shit of the whole world}}
\end{center}

\input{resumen_eng.tex}

\chapter*{Agradecimientos}
\input{agradecimientos.tex}
\addcontentsline{toc}{chapter}{Agradecimientos}

\chapter*{Acrónimos}
\begin{acronym}
\acro{AAO} {Oscilación Antártica}, en inglés.

\acro{AMO} {Oscilación multidecadal del Atlántico}, en inglés.

\acro{ATSM}{Anomalías de la \acl{TSM}}.

\acro{AVHRR}{Advanced Very-High-Resolution Radiometer}

\acro{BaRDO}{Base Regional de Datos Oceanográficos}

\acro{CAS}{Corriente del Atlántico Sur}.

\acro{CBM}{Confluencia Brasil-Malvinas}

\acro{CCA}{Corriente Circumpolar Antártica}.

\acro{CFSR}{Climate Forecast System Reanalysis}

\acro{CM}{Corriente de Malvinas}.

\acro{CCMPv2}{Cross-Calibrated Multi-Platform gridded surface vector winds product, version 2}, en inglés.

\acro{ENOS}{El Niño Oscilación del Sur}.

\acro{EOF}{Funciones Ortogonales Empíricas}

\acro{GTS}{Sistema Global de Telecomunicaciones}, en inglés.  

\acro{INIDEP}{Instituto Nacional de Desarrollo Pesquero}

\acro{MK}{Mann Kendall}

\acro{MDT} {Topografía Dinámica Media}, en inglés.

\acro{MTM} {Análisis espectral de Multitaper}, en inglés.

\acro{NCEPR1}{National Centers for Environmental Prediction Reanalysis 1}, en inglés.

\acro{NOAA}{National Oceanic and Atmospheric Administration}.

\acro{OISSTv2}{Optimally Interpolation SST Version 2}.

\acro{ONI}{Oceanic Niño Index}

\acro{PC} {Componente principal}, en inglés.

\acro{PCA} {Análisis de Componentes Principales}, en inglés.

\acro{PDO} {Oscilación Decadal del Pacífico}, en inglés.

\acro{PNM}{Presión a Nivel del Mar}

\acro{PP}{Plataforma Patagónica}.

\acro{PPN}{\acl{PP} Norte}.

\acro{PPS}{\acl{PP} Sur}.

\acro{SAM}{Índice Anular del Sur}, en inglés.

\acro{SVD}{Descomposición de valores singulares}

\acro{SWAO}{Océano Atlántico Sudoccidental}, en inglés. 

\acro{TSM}{Temperatura Superficial del Mar}.

\acro{WOD}{World Ocean Database}.
\end{acronym}

\addcontentsline{toc}{chapter}{Acrónimos}

\chapter*{Publicaciones}
\addcontentsline{toc}{chapter}{Publicaciones}

\begin{itemize}%
\item {Artículos en revistas internacionales con referato:}

\begin{itemize}%
	
\item \textbf{Silicon Isotope Separation by two frequency IRMPD}, M. Risaro, V. D'Accurso, J. Codnia, M.L. Azc\'arate. 
\textit{Sociedad Española de Óptica; \'Optica Pura y Aplicada}; 50; 3; 8-2017; 229-237.	
	
\end{itemize}

%	\item{Art\'iculos en revistas nacionales con referato}
%	\begin{itemize}%
	
%	\item \textbf{Desarrollo de un l\'aser TEA de CO$_{2}$ de alta coherencia.}, M. Risaro, J. Codnia, M. L. Azcárate. \textit{Anales AFA} 25; 4; 11-2014; 167-170.
	
%	\end{itemize}

\end{itemize}

%% INDICE!
\tableofcontents
\listoffigures
\listoftables

%%%%%%%%%%%%%%%%%%%%%%%%%%%%%%%%%%%%%%%%%%%%%%%%%%%%%%%%%%%%%%%%%%%%%%%%%%%%%%%%%%%%%%%%%%%%%
\mainmatter
%%%%%%%%%%%%%%%%%%%%%%%%%%%%%%%%%%%%%%%%%%%%%%%%%%%%%%%%%%%%%%%%%%%%%%%%%%%%%%%%%%%%%%%%%%%%%
\chapter{Introducción}
\label{chap:intro}

no se sabe una mierda, y yo quiero saber algo.


%%%%%%%%%%%%%%%%%%%%%%%%%%%%%%%%%%%%%%%%%%%%%%%%%%%%%%%%%%%%%%%%%%%%%%%%%%%%%%%%%%%%%%%%%%%%%%%%
\chapter{Datos}
\label{chap:2}

%%%%%%%%%%%%%%%%%%%%%%%%%%%%%%%%%%%%%%%%%%%%%%%%%%%%%%%%%%%%%%%%%%%%%%%%%%%%%%%%%%%%%%%%%%%%%%%%
\chapter{Metodología}
\label{chap:3}


%%%%%%%%%%%%%%%%%%%%%%%%%%%%%%%%%%%%%%%%%%%%%%%%%%%%%%%%%%%%%%%%%%%%%%%%%%%%%%%%%%%%%%%%%%%%%%%%
\chapter{Resultados?}
\label{chap:4}

%%%%%%%%%%%%%%%%%%%%%%%%%%%%%%%%%%%%%%%%%%%%%%%%%%%%%%%%%%%%%%%%%%%%%%%%%%%%%%%%%%%%%%%%%%%%%%%%
\chapter{Resultados pedorros 2}
\label{chap:5}

%%%%%%%%%%%%%%%%%%%%%%%%%%%%%%%%%%%%%%%%%%%%%%%%%%%%%%%%%%%%%%%%%%%%%%%%%%%%%%%%%%%%%%%%%%%%%%%%
\chapter{Covariabilidad entre la temperatura y la presion}
\label{chap:6}


%%%%%%%%%%%%%%%%%%%%%%%%%%%%%%%%%%%%%%%%%%%%%%%%%%%%%%%%%%%%%%%%%%%%%%%%%%%%%%%%%%%%%%%%%%%%%%%%
\chapter{Resultados}
\label{chap:7}

%%%%%%%%%%%%%%%%%%%%%%%%%%%%%%%%%%%%%%%%%%%%%%%%%%%%%%%%%%%%%%%%%%%%%%%%%%%%%%%%%%%%%%%%%%%%%%%%
\input{chap_08_dani.tex}

%%%%%%%%%%%%%%%%%%%%%%%%%%%%%%%%%%%%%%%%%%%%%%%%%%%%%%%%%%%%%%%%%%%%%%%%%%%%%%%%%%%%%%%%%%%%%%%%
\part*{Apéndices}
\addcontentsline{toc}{chapter}{Apendices}
\appendix
\renewcommand*{\chaptername}{\appendixname}

\chapter{Apéndice 1}
\label{chap:apendice}


\backmatter


\printbibliography
\end{document}


